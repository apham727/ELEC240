\documentclass[10pt]{article}

\usepackage{lipsum}
\usepackage{url}
\usepackage{float}
\usepackage{amsmath}
\usepackage{enumitem}
\usepackage{graphicx}
\usepackage{caption}
\usepackage{subcaption}
\usepackage{rotating}
\usepackage{geometry}
\usepackage{listings}
\usepackage{hyperref}
\usepackage[T1]{fontenc}
\usepackage[numbered]{matlab-prettifier}

\newcommand{\documentTitle}{Lab 2 - Signal Sources and Sinks}
\newcommand{\documentAuthor}{Andrew Pham, Aneel Damaraju}
\newcommand{\courseTitle}{ELEC 240}
\newcommand{\testDate}{September 11, 2018}
\newcommand{\reportDate}{September 18, 2018}

\geometry{margin=1in}
\lstset{
    tabsize=4,
    basicstyle={\ttfamily},
    captionpos=b,
    belowskip=1em,
    aboveskip=1em,
    numbers=left,
	escapechar=\@,
}

\title{
    \textbf{\courseTitle} \\
    \textbf{\documentTitle} \\
    \bigskip
    \textbf{\large{Test performed: \testDate}} \\
    \textbf{\large{Report submitted: \reportDate}} \\
    \bigskip
    \bigskip
}
\author{\documentAuthor}
\date{}

\begin{document}

\maketitle

\newpage

\section{Objective}

In this lab, we explored how to detect and change signals through the use of electroacoustic transducers. In the first section, we observed the various properties of a signal such as its frequency and amplitude through a speaker and calculated how measuring this signal with the speaker affects the circuit. In the second section, we now produced signals through various methods like our vocal cords and the lab PC and viewed these signal properties on the oscilloscope. Finally, in section three, we independently viewed the signal properties of a photodiode and a light-emitting diode, then combined the two to view how a light-emitting diode can send signals to a photoresistor to achieve optoelectronic communication. 

\medskip

%\textit{Note (To be deleted): Think of this test report as a document with your peers as your readers. This means you can assume a similar knowledge background as you. Your readers should be able to easily understand what is going on, and also be able to repeat your lab results based on your document and all references you cite.}



%\textit{For the Objective section, identify the test you performed and its objectives. The objectives of the test are important to state because they are usually analyzed in the conclusion to determine whether the test succeeded.}

\section{Materials}

\begin{itemize}
	\item Virtual Bench (Software, Oscilloscope, Function Generator, DC Power Supply)
	\item BNC Male to Clips cord
	\item Oscilloscope Probe
	\item Speaker
	\item Breadboard
	\item Microphone
	\item 2 10 cm length wires (with 6 mm stripped on each end)
	\item Lab PC with associated sound files and sound card cable
	\item Photodiode 
	\item Red LED
	\item BNC Banana Adapter
	\item Digital Multimeter
	\item 220 Ohm Resistor
\end{itemize}

\medskip

%\textit{Note (To be deleted): Provide a bullet point list of components, software tools, and hardware (such as the NI VirtualBench or DMM) used during the lab}

\section{Test Description}

\subsection{Electroacoustic Transducers I}
We began by creating a 1kHz sine wave, connecting this signal to a speaker, and listening to the signal in audio form. We then varied the parameters of the input signal to listen to the effects on the output signal. Then using these measurements, we found the Thevenin voltage, and used this value to attempt to find the maximum power transfer.

\subsection{Electroacoustic Transducers II}
This time, we varied the input of the oscilloscope by attaching it to a microphone. After making various sounds into the microphone and visualizing the signals, we attempted to qualitatively analyze the signals created by vowels sounds created by humans as well as a virtual piano. After this, we attempted to analyze various given audio signals through the use of a speaker and the oscilloscope attached to the sound card cable of the computer playing the sounds.

\subsection{Optoelectrical Signal Sources and Sinks}
We created a simple circuit connecting a photodiode to the oscilloscope and observed the AC and DC signals created by this circuit before and after covering the diode with a hand. After this a LED was connected to the DC power supply. and qualitative observations were recorded while varying the amplitude and frequency of the incoming voltage. Then the optical source and sink were combined, and the measurement across the photodiode was read as it received signals from blinking LED.

\medskip

%\textit{Note (To be deleted): This section provides a summary of the test your team performed. Give enough information so readers can understand what you did, but do not go into the details of every step.}

\subsection{Pre-Lab Calculations and Schematics}

No pre-lab calculations were needed, but 

\medskip

\textit{Note (To be deleted): Include the homework pre-calculations and schematics that serve as the initial setup for the test. Briefly explain the importance of each item you include. You may want to number your equations/figures so you can refer to them in later sections. Including photos of handwritten work is okay.}

\section{Results and Discussion}

Your text here

\medskip

\textit{Note (To be deleted): The heart of your report is the presentation of your results and a discussion of those results. In your discussion, you should not only analyze your results, but also discuss the implications of those results.}

\section{References}

Your text here

\medskip

\textit{Note (To be deleted): List any datasheets, websites, lab procedure, etc. used during the lab.}

\section{Conclusion}

Your text here

\medskip

\textit{Note (To be deleted): While the ``Results and Discussion'' section focused on the test results individually, the ``Conclusion'' discusses the results in the context of the entire experiment. Usually, the objectives given in the ``Introduction'' are reviewed to determine whether the experiment succeeded. If the objectives were not met, you should analyze why the results were not as predicted.}

\section{Errors}

Your text here

\medskip

\textit{Note (To be deleted): Briefly list sources of error and discuss how to eliminate or deal with them}

\end{document}
