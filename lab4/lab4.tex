\documentclass[10pt]{article}

\usepackage{lipsum}
\usepackage{url}
\usepackage{float}
\usepackage{amsmath}
\usepackage{enumitem}
\usepackage{graphicx}
\usepackage{caption}
\usepackage{subcaption}
\usepackage{rotating}
\usepackage{geometry}
\usepackage{listings}
\usepackage{hyperref}
\usepackage[T1]{fontenc}
\usepackage{siunitx}
\usepackage[numbered]{matlab-prettifier}

\newcommand{\documentTitle}{Lab 1 - Op-amps}
\newcommand{\documentAuthor}{John Doe, Mary Smith}
\newcommand{\courseTitle}{ELEC 244}
\newcommand{\testDate}{January 17, 2018}
\newcommand{\reportDate}{January 24, 2018}

\geometry{margin=1in}
\lstset{
    tabsize=4,
    basicstyle={\ttfamily},
    captionpos=b,
    belowskip=1em,
    aboveskip=1em,
    numbers=left,
	escapechar=\@,
}

\title{
    \textbf{\courseTitle} \\
    \textbf{\documentTitle} \\
    \bigskip
    \textbf{\large{Test performed: \testDate}} \\
    \textbf{\large{Report submitted: \reportDate}} \\
    \bigskip
    \bigskip
}
\author{\documentAuthor}
\date{}

\begin{document}

\maketitle

\newpage

\section{Objective}

In this lab, we worked on understanding the workings of both breadboards and Op-Amps. We learned how to wire and set up a breadboard for future use, a well as initialize an Op-Amp in your circuit. Experimental values were compared to calculated ones through the use of the oscilloscope. We learned how to create an inverting amplifier and how the values are gated by the DC offset as well as the power supply. Fiially,zen preparation for the lab next week, we assembled a circuit containing two microphones and a speaker in order to transmit and amplify external signals coming in the form of audio waves.

\section{Materials}
\begin{itemize}
	\item Breadboard
	\item 741 Op Amp
	\item NI Virtual Bench
	\item \SI{10}{\ohm} Resistor
	\item \SI{100}{\ohm} Resistor
	\item \SI{10}{\kilo\ohm} Resistor (2)
	\item \SI{100}{\kilo\ohm} Resistor
	\item \SI{.1}{\micro\farad} Capacitor
	\item Dynamic Microphone
	\item Telephone handset
\end{itemize}


\section{Test Description}
\subsection{The 741 Op Amp}
After learning about the Op Amp, we compared open and closed loop configurations, and understood the concepts of DC offset amplification and clipping. By creating simple circuits in this format, we familiarized ourselves with the pins on the 741 Op Amp, and creating functional circuits that would not blow out the device.
\subsection{The Inverting Configuration}
In this smaller section of the lab, we focused on the functions of closed loop inverting amplifier configurations of the Op Amp. We learned about topics such as clipping and the slew rate, and how the internal properties of the Op Amp effect the output signal in comparison to the expected output.
\subsection{Transducer Amplifiers}
As the name implies, in this section we used what we knew about Op Amps to create an amplifier that worked on the audio signals created by a dynamic microphone. These signals were first viewed by the Virtual Bench, until the voltage was amplified to an acceptable level. Then, this circuit was attached to the circuit for a phone handset, to show that the Op Amp amplified the audio signal as well. 

\subsection{Pre-Lab Calculations and Schematics}

Before the lab, there we no calculations to be done, just some setup for the breadboard. To power all of the terminals we needed to set up the power bus, by connecting it to all 3 of the power rails as seen in Fig \ref{Fig: Breadboard}. After the power was properly connected, we noted the proper connections for the Interface Board connectors, which would be needed in the lab.

\section{Results and Discussion}



\section{References}

https://www.ece.rice.edu/~dpr2/elec240/lab4/

\section{Conclusion}

\subsection{4.1: The 741 Op Amp}
\subsubsection{Part A: Powering Up the 741 Op Amp}
The first part of this lab was simply connecting the Op Amp to the breadboard and understanding the pins on the Op Amp. The 
\section{Errors}

Your text here

\medskip

\textit{Note (To be deleted): Briefly list sources of error and discuss how to eliminate or deal with them}

\end{document}
