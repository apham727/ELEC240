\documentclass[10pt]{article}

\usepackage{lipsum}
\usepackage{url}
\usepackage{float}
\usepackage{amsmath}
\usepackage{enumitem}
\usepackage{graphicx}
\usepackage{caption}
\usepackage{subcaption}
\usepackage{rotating}
\usepackage{geometry}
\usepackage{listings}
\usepackage{hyperref}
\usepackage[T1]{fontenc}
\usepackage[numbered]{matlab-prettifier}

\newcommand{\documentTitle}{Final Project - RF Transceiver System}
\newcommand{\documentAuthor}{Andrew Pham, Aneel Damaraju}
\newcommand{\courseTitle}{ELEC 240}
\newcommand{\testDate}{November 6, 2018}
\newcommand{\reportDate}{December 1, 2018}

\geometry{margin=1in}
\lstset{
    tabsize=4,
    basicstyle={\ttfamily},
    captionpos=b,
    belowskip=1em,
    aboveskip=1em,
    numbers=left,
	escapechar=\@,
}

\title{
    \textbf{\courseTitle} \\
    \textbf{\documentTitle} \\
    \bigskip
    \textbf{\large{Test performed: \testDate}} \\
    \textbf{\large{Report submitted: \reportDate}} \\
    \bigskip
    \bigskip
}
\author{\documentAuthor}
\date{}

\begin{document}

\maketitle

\newpage

\section{Objective}

Our goal in this project is to construct an RF transceiver system that can transmit and receive audio signals across the 160-190kHz band. There are three major components to the transmitter: the analog input circuitry, the digital signal processing 

\medskip

\textit{Note (To be deleted): Think of this test report as a document with your peers as your readers. This means you can assume a similar knowledge background as you. Your readers should be able to easily understand what is going on, and also be able to repeat your lab results based on your document and all references you cite.}

\textit{For the Objective section, identify the test you performed and its objectives. The objectives of the test are important to state because they are usually analyzed in the conclusion to determine whether the test succeeded.}

\section{Materials}

Your text here

\medskip

\textit{Note (To be deleted): Provide a bullet point list of components, software tools, and hardware (such as the NI VirtualBench or DMM) used during the lab}

\section{Test Description}

Your text here

\medskip

\textit{Note (To be deleted): This section provides a summary of the test your team performed. Give enough information so readers can understand what you did, but do not go into the details of every step.}

\subsection{Pre-Lab Calculations and Schematics}



\medskip

\textit{Note (To be deleted): Include the homework pre-calculations and schematics that serve as the initial setup for the test. Briefly explain the importance of each item you include. You may want to number your equations/figures so you can refer to them in later sections. Including photos of handwritten work is okay.}
===
\section{Results and Discussion}

\subsection{Construct subsystems}

\subsubsection{Analog Signal Generation}

This section uses the circuit constructed in previous labs, consult Figure \ref{fig:asg}. In short, this op amp circuit connects to the different 

\subsubsection{Analog Speaker Driver}

As the name implies, this Op-Amp circuit connects the speaker to a $V_{in}$. In this case, the input voltage is an analog sound that will be sent over the radio. 

\subsubsection{Building Labview VIs}

Two Labview VIs were built, the transmitter and the receiver. The transmitter function is created with the output $x(t)  = A (1 + m(t)) sin(2 \pi f_c t)$ where the message signal is input from the breadboard through the DAQ cable and the transmitted signal is output from the DAQ back to the breadboard, and to the RF module, to be transmitted. 

The reciever is created to take in the previous modulated signal and then further modulate it by $sin(2 \pi f_c t)$ and then lowpass filter this signal at the bandwidth of speech in order to remove the higher frequency modulations that result. 

\subsubsection{Characterizing modules with Bode Plots}

\subsection{Self Test}
	


\medskip

\section{References}

Your text here

\medskip

\textit{Note (To be deleted): List any datasheets, websites, lab procedure, etc. used during the lab.}

\section{Conclusion}

Your text here

\medskip

\textit{Note (To be deleted): While the ``Results and Discussion'' section focused on the test results individually, the ``Conclusion'' discusses the results in the context of the entire experiment. Usually, the objectives given in the ``Introduction'' are reviewed to determine whether the experiment succeeded. If the objectives were not met, you should analyze why the results were not as predicted.}

\section{Errors}

Your text here

\medskip

\textit{Note (To be deleted): Briefly list sources of error and discuss how to eliminate or deal with them}

\end{document}
