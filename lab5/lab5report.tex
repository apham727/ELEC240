\documentclass[10pt]{article}

\usepackage{lipsum}
\usepackage{url}
\usepackage{float}
\usepackage{amsmath}
\usepackage{enumitem}
\usepackage{graphicx}
\usepackage{caption}
\usepackage{subcaption}
\usepackage{rotating}
\usepackage{geometry}
\usepackage{listings}
\usepackage{hyperref}
\usepackage[T1]{fontenc}
\usepackage[numbered]{matlab-prettifier}

\newcommand{\documentTitle}{Lab 5 - Signal Analysis and Characterization}
\newcommand{\documentAuthor}{Andrew Pham, Aneel Damaraju}
\newcommand{\courseTitle}{ELEC 240}
\newcommand{\testDate}{October 2, 2018}
\newcommand{\reportDate}{October 16, 2018}

\geometry{margin=1in}
\lstset{
    tabsize=4,
    basicstyle={\ttfamily},
    captionpos=b,
    belowskip=1em,
    aboveskip=1em,
    numbers=left,
	escapechar=\@,
}

\title{
    \textbf{\courseTitle} \\
    \textbf{\documentTitle} \\
    \bigskip
    \textbf{\large{Test performed: \testDate}} \\
    \textbf{\large{Report submitted: \reportDate}} \\
    \bigskip
    \bigskip
}
\author{\documentAuthor}
\date{}

\begin{document}

\maketitle

\newpage

\section{Objective}

The first objective of this lab was to learn how to generate a virtual signal with LabView and vary the signal's parameters. The second objective of the lab was to learn how to acquire a signal in LabView from an external source like our VirtualBench function generator and then perform frequency analysis on the acquired signal. The third and final objective of the lab was to modify the circuit from Lab 4 so that we could perform speech analysis as well as analyze an unknown signal. 

\medskip

%\textit{Note (To be deleted): Think of this test report as a document with your peers as your readers. This means you can assume a similar knowledge background as you. Your readers should be able to easily understand what is going on, and also be able to repeat your lab results based on your document and all references you cite.}

%\textit{For the Objective section, identify the test you performed and its objectives. The objectives of the test are important to state because they are usually analyzed in the conclusion to determine whether the test succeeded.}

\section{Materials}

\begin{itemize}
	\item Virtual Bench (Software, Oscilloscope, Function Generator, DC Power Supply)
	\item LabView
	\item BNC Male to Clips cord
	\item BNC T connector
	\item Oscilloscope Probe
	\item Breadboard (with setup from Lab 4
	\item 2 10 cm length wires (with 6 mm stripped on each end)
	\item Digital Multimeter
	\item 2.2 $k\Omega$ resistor
	\item 033 $\mu F$ capacitor
	\item Telephone handset
	\item Dynamic microphone
	\item Smartphone (or some device to play audio from a speaker)
	
\end{itemize}

\medskip

%\textit{Note (To be deleted): Provide a bullet point list of components, software tools, and hardware (such as the NI VirtualBench or DMM) used during the lab}

\section{Test Description}

In Experiment 5.1, Part A, we focused on generating a signal in Labview by configuring circuit components on the block diagram pane in Labview. Once we created a configuration to generate a circuit, we created a spectrum analyzer in Part B by adding a Fast Fourier Transformer to the configuration and observed how varying the parameters of both the signal and measurement tools could affect the power spectrum graph. 

In Experiment 5.2, Part A, we acquired a signal from the NI VirtualBench by connecting the Data Acquisition Card (DAQ) within the PC running the LabView software via a DAQ cable. In this step, we also had to modify the previous configuration from Experiment 5.1 so that the circuit would accept an external signal. We achieved this by adding a DAQ assistant block so that LabView would read the incoming signal from the DAQ cable. In Part B, we explored the spectrum of triangle waves and how it varied as we tweaked the symmetry of the triangle wave. We also explored the spectrum of square waves and how it varied as we tweaked the duty cycle of the square wave. In Part C, we observed the spectrum and frequency response of a low-pass RC circuit by testing the circuit response at various frequencies. 

In Experiment 5.3, Part A, we explored the spectrum of speech signals by observing the spectral graph generated by making various vowel sounds and whistling with our mouths. By observing the spectra of these various signals, we estimated the approximate bandwidth for speech. In Part B, we modified the sound circuit from Lab 4 to spectrally analyze the Mystery Signal to determine how it achieves the effect of sounding like it was always ascending. 

\medskip

%\textit{Note (To be deleted): This section provides a summary of the test your team performed. Give enough information so readers can understand what you did, but do not go into the details of every step.}

\subsection{Pre-Lab Calculations and Schematics}

No pre-lab calculations were necessary; however, we needed an understanding of decibels and their relation to power in order to perform the experiments in the lab. The standard definition of power ration in decibels is as follows:

$$ \text{power ratio in decibels} = 10log(\frac{P_1}{P_0})$$

Since power across a load resistor can be defined as $$P = V^2/R$$then we can re-express the power ratio equation in a more convenient form as follows: $$\text{power ratio in decibels} = 10log(\frac{V_1^2/R_L}{V_0^2/R_L}) = 20log(\frac{V_1}{V_0})$$

With this expression of a power ratio, we can more conveniently express the ratio of sound pressure levels as $$\text{SPL} = 20log(\frac{p}{p_0})$$ where $p_0$ is the reference sound pressure 

\medskip

%\textit{Note (To be deleted): Include the homework pre-calculations and schematics that serve as the initial setup for the test. Briefly explain the importance of each item you include. You may want to number your equations/figures so you can refer to them in later sections. Including photos of handwritten work is okay.}

\section{Results and Discussion}

Your text here

\medskip

\textit{Note (To be deleted): The heart of your report is the presentation of your results and a discussion of those results. In your discussion, you should not only analyze your results, but also discuss the implications of those results.}

\section{References}

Your text here

\medskip

\textit{Note (To be deleted): List any datasheets, websites, lab procedure, etc. used during the lab.}

\section{Conclusion}

Your text here

\medskip

\textit{Note (To be deleted): While the ``Results and Discussion'' section focused on the test results individually, the ``Conclusion'' discusses the results in the context of the entire experiment. Usually, the objectives given in the ``Introduction'' are reviewed to determine whether the experiment succeeded. If the objectives were not met, you should analyze why the results were not as predicted.}

\section{Errors}

Your text here

\medskip

\textit{Note (To be deleted): Briefly list sources of error and discuss how to eliminate or deal with them}

\end{document}
